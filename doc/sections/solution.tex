\section{Lösungsidee}\label{sec:idea}
\begin{flushleft}
Die Idee besteht darin zu erst das Bild einzulesen um Zugriff auf die RGB-Werte der einzelnen Pixel zu erhalten.\\
Man beginnt mit dem Pixel oben links (Aufgabenstellung).\\
Nun muss nach Aufgabenstellung der R-Wert ausgelesen werden.\\
Dieser wird in ein ASCII Zeichen umgewandelt und an die Gesamtzeichenkette angehangen.\\
Nun wird der G- und B- Wert ausgelesen und als x bzw.\ y Verschiebung des neuen Pixels genommen.\\

Die x Kooridnate lässt sich nun über $new\_x = (old\_x + x\_move)\%picture\_width$ berechnen
Gleiches gilt für y nur das dabei modulo picture\_height gerechnet werden muss.
Durch berechnen des modulo wird erreicht, dass x bzw.\ y nie größer sind als das echte Bild und die Verschiebung richtig berechnet wird.\\
Nun wird das Pixel an der Stelle (x|y) abgerufen.\\
Mit diesem Pixel wiederholt man nun den Ablauf.\\
Das Ende hat man erreicht wenn $G = B = 0$ gilt.\\
Als letztes wird die nun finale Zeichenkette ausgegeben.\\
\end{flushleft}
