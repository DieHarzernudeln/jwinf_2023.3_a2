\section{Umsetzung}
Die Umsetzung dieser Aufgabe erfolgte in Rust, einer Programmiersprache, die für ihre Sicherheit, Geschwindigkeit und Nebenläufigkeit bekannt ist.

\subsection{Einlesen der Eingabedaten}
Die Funktion \texttt{read\_data()} liest das Bild als \texttt{DynamicImage} aus einer Datei deren Pfad als Argument vorgegeben ist oder von der Benutzereingabe stammt.
Das DynamicImage gibt z.B. Breite, Höhe und Pixel in eine gut verarbeitbare Form.\\
Dies geschieht über die Nutzung der \texttt{image} Bibliothek von Rust, welche sich für das Laden von allen gängigen Bildformaten eignet.\\
Dokumentation und SourceCode der Bibliothek: \url{https://docs.rs/image/latest/image/}

\subsection{Analysieren der Pixel}\label{sec:pixel analysis}

Das Analysieren der Pixel geschieht in der Funktion \texttt{analyze\_pixel(pixel: Rgba<u8>) -> (char, u32, u32)}\\
Dabei wird der R-Wert über eine simple \texttt{as} Konvertierung in ein ASCII Zeichen umgewandelt\\
(\texttt{pixel[0] as char})
in dem der u8 Wert als ASCII Zeichenindex interpretiert wird.\\
Die G (pixel[1]) und B (pixel[2]) Werte werden zusammen mit dem Zeichen als dreigliedriges Tupel zurückgegeben

\subsection{Algorithmus}
Der Algorithmus beginnt damit, dass man den Output-String \texttt{out} definiert und initialisiert.
Nun geht man wie in~\ref{sec:idea}.\ beschrieben vor und startet oben links, wobei x = 0 und y = 0 gilt.\\
\linebreak
Folgendes geschieht nun solange $x \neq 0$ und $y \neq 0$:
\begin{adjustwidth}{10px}{}
\begin{itemize}
\item Das aktuelle Pixel wird anhand der Koordinaten aus dem Bild abgerufen
\item Das aktuelle Pixel wird nun siehe~\ref{sec:pixel analysis} analysiert.
\item Nun wird das neue Zeichen an den Output-String angehangen. $\rightarrow$ \texttt{out.push(ch);}
\item Als letztes wird wie in~\ref{sec:idea}.\ die x bzw.\ y Koordinate des neuen Pixels berechent.
\subitem{Die Höhe bzw.\ Breite des Bildes stammt dabei vom Einlesen der Daten}
\end{itemize}
\end{adjustwidth}
\vspace*{0.5cm}
Schlussendlich wird noch der Gesamt-String \texttt{out} ausgegeben.